% In tal senso negli ultimi anni è andato sviluppandosi un nuovo paradigma di
% calcolo detto Service Oriented Computing (SOC), in cui i “mattoni” utilizzati
% per costruire le applicazioni sono chiamati servizi. 
% Un servizio non è altro che un modulo che assolve specifici compiti ed è
% corredato da una descrizione formale delle sue 
% funzionalità, realizzata secondo lunguaggi e formalismi descritti da specifiche
% pubbliche (standard). La composizione di tali servizi consente quindi uno
% sviluppo veloce ed economico di nuove applicazioni e gioca un ruolo importante nell’integrazione dei sistemi, sia all’interno del singolo dominio aziendale che in ambito inter-aziendale.

% La descrizione formale di un servizio permette di rendere pubblica la
% sua interfaccia, cioè l’insieme di operazioni che il servizio può eseguire, ed                                                                    
% il suo comportamento, cioè gli effetti di tali operazioni. Oltre a tali aspetti                          
% strettamente funzionali, la descrizione di un servizio può esprimere anche
% aspetti legati alla qualità del servizio, come ad esempio requisiti di
% affidabilità, efficienza (tempi di risposta) e sicurezza (garanzia sulla
% confidenzialità delle a informazioni). La possibilità di reperire e consultare
% tali descrizioni sta alla base dei meccanismi che consentono di individuare, scegliere e comporre i
% servizi allo scopo di costruire le proprie applicazioni. Si veda [PG03] per una
% breve introduzione al SOC.

% Avere a disposizione gli strumenti per reppresentare i dati in maniera
% totalmente portabile e sicura, e accedere alle risorese secondo un modello di
% elaborazione formale altamente produttivo e multipiattaforma, non esauirsce
% tutte le necessità.
