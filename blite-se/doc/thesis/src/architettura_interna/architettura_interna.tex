\chapter{Architettura interna}

In questo capitolo illustriamo l'architettura interna dell'Engine, cioè del
componente software predisposto alla messa in esecuzione dei programmi Blite.

Ricordiamo che nello scenario delineato avremo un Engine per locazione, o se si
vuole per nodo di rete, e su ognuno di questi componenti sarà possibile
installare o rimuovere definizioni di processi Blite. In pratica un engine
gestir\`a un insieme di definizioni, creando da queste istanze di
processi e utilizzerà l'Environment per interagire con gli altri Engine. 
Dall'environment stesso l'engine verr\`a notificato riguardo l'accedere di
eventi quali l'arrivo di messaggi indirizzati alle porte delle sue definizioni.

Dal punto di vista logico relazionale abbiamo gi\`a individuato le seguenti
macro entit\`a e realazioni

[\ldots immagine]

Prima di entrare nel dettaglio delle scelte architetturali ricapitoliamo quale
sono le caratteristiche peculiari di un sistema che deve gestire programmi per
l'orchestrazione di servizi in modo che sia pi\`u facile da una parte
comprendere e dall'altra giustificare le scelte fatte.

A nostro vantaggio:
\begin{itemize}
  \item Un Engine contine in genarale un numero contenuto di definizioni, per
  cui non ci interessa la scalibilit\`a rispetto alla quantita di definizioni
  istallate su singolo engine. Tale scalabilit\`a al contrario pu\`o essere
  ottenuta aggiungendo altri engine e installando definizioni su engine diversi.
  
  \item  Una definizione (o programma) Blite avendo pricipalmete funzinalit\`a
  di integrazione avr\`a una lunghezza generalmente limitata. 
  
  \item Poich\`e le operazione fondamentali di un programma di questo genere
  sono invocazioni remote, le durate delle esucizioni hanno ordine di grandezza
  dettati dai tempi caratteristici della rete. Per questo motivo non risulta
  determinante l'efficenza di esecuzione delle operazioni interne di un
  processo. Il nostro engine non necessiter\`a una particolare ottimizzazione 
  rispetto all'efficenza di esecuzione interna.
\end{itemize}

al contrario risultano perticolarmente critici i seguenti aspettti:
\begin{itemize}
  \item Per ciascuna definizione potr\`a essere richiesto la creazionee di
  innumerevoli richieste. La scalabilit\`a rispetto al numero di instanze di
  processo e quindi al numero delle richieste remote risulta essere un
  prerequisito fondamentale.
  
  \item Se da un lato abbimo detto 
\end{itemize}

 

